
\section{Conclusion and Future Work}

Given the data units, access requirements, and the desired upper bound on
redundancy factor, we have proposed an algorithm that would create a cache
oblivious layout with the primary goal of reducing the seek time through
duplicating the data units. We proposed a cost model for estimating the seek
time, and in our algorithm we can move or copy data units in appropriate
locations such that it reduces the estimated seek time.  We have shown that
such a layout significantly improves both the performance and consistency of
interactivity in massive model walkthrough applications.  \\ \\
Our proposed redundant storage of data may limit editing and modification of data because the data has to be modified at all copies. However, we foresee no problem in recomputing and updating the layout due to this modification using our algorithm since every iteration in our algorithm just assumes a layout and improves on it. After data modification, we can delete/modify the relevant data units, update the access pattern and run a few iterations of our algorithm to get a better layout. In other words, our algorithm is incremental and can be used for dynamic data sets, which also might be a result of scene editing and modification.\\
\\
{\bf Limitations and future work:}
Our cost model does not take account distance between access requirements. We only take into account distance between data units in the same access requirement and we do not consider the seek time between access requirements. If we do take this account in our model, then if we are given information as to which access requirement is more likely to be used before or after another access requirement we would have an even more accurate model for seek time.

