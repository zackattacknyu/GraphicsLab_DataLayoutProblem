\section{Complexity Analysis}

We now analyze the running time and storage requirements of our algorithm. Let $N$ be the number of data units and $A$ be the number of access requirements. We will use $k$ as the average span of a single access requirement. The variable $Q$ will represent the number of data units that can be copied as specified by the user. Let $r$ be the amount of redundancy if we have a single-seek layout \cite{singleseeklayout}. The average number of overlapping access requirements for a single data unit ends up being $O(rN)$. 

{\bf Time Complexity:} The construction of the heaps $E_M$ and $E_C$ involve computing the benefit information for all $A$ access requirements and inserting each one into the heap. Computing the benefit information of moving or copying a data unit involves scanning the span of the access requirement. We justify this approach which takes $O(k)$ operations in Appendix A. Inserting this benefit information into the heap takes $O(log (A))$ operations. Thus it takes a total of $O(A (k + logA))$ operations to do the initial construction of the heaps. \\
\\
After the initial construction, in every iteration, an element from the top of the heap is removed and processed, the benefit function is recalculated for affected access requirements, and the heap is updated. For each data unit move or copy, $O(rN)$ overlapping access requirements are affected, and for each of these access requirements $O(k+log(A))$ is required to recalculate the benefit data and update the heap. Thus each iteration takes $O(rN(k + logA))$ operations.\\
\\
For simplicity we will assume that the loop where we move the data units instead of copying is run $O(N)$ times total. This comes from the fact that the cache oblivious layout \cite{cacheobliviouslayout} should be a good approximation so the number of moves that would still be useful should be limited. We defined $Q$ as the number of redundant data units added. We thus can assert that there are $O(Q + N)$ iterations of the move or copy then update operation. This means that the moving and copying loops will take a total of $O((QrN + rN^2)(k + logA))$ operations. \\
\\
In total, our algorithm takes $O((QrN + rN^2 + A)(k + logA))$ operations. In practice, the optimal redundancy factors we found were greater than $2$ thus we can say that $Q > N$ meaning that $QrN > rN^2$. Additionally, since $r \geq 1$, we know that $rN > > A$. Thus we can simplify the expression to say that our algorithm takes $O(QrN(k + logA))$ operations. \\
\\
{\bf Space Complexity} During the run of the algorithm, we have to store the number of overlapping access requirements at each data unit, which will require $O(N)$ storage. We will also have to store a heap of access requirements, which can be stored using $O(A)$ space. We also have a list of access requirements and that information will take up $O(A)$ space. In total we thus have $O(A + N)$ storage space used during the run of the algorithm. 