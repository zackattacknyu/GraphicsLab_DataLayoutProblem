
\section{Introduction}

In typical walkthrough systems, data sets consisting of hundreds of millions of
triangles and many gigabytes of associated data (e.g. walking through a virtual
city) are quite common. Rendering such massive amounts of data requires
out-of-core rendering algorithms that bring only the required data for
rendering into main memory from secondary storage. The time required to accomplish such a task
is known as seek time for our purposes and it
is the time taken to locate the beginning of the required data in the storage
device. \\
\\
In this paper, we store redundant
copies of data in order to reduce the seek time. We propose a model for seek time based on the actual
number of data units
between the requested data units in the linear data layout. Using this model, and given
the data access requirements for a walkthrough application, we develop
an algorithm to duplicate data units strategically to maximize the reduction
in the seek time, while keeping the redundancy factor within the user defined
bound. We show that our greedy solution can generate both the extreme cases
of data layout with redundancy, namely the maximum redundancy case
(a layout where seek time is at most one) and the no-redundancy case (a simple
cache oblivious mesh layout with a potentially high seek time), as well as
reasonable solutions for redundancy factor constraints in between the extremes.
We show that the
implementation of our algorithm significantly reduces average delay and the maximum delay between
frames and noticeably improves the consistency of performance and
interactivity.

