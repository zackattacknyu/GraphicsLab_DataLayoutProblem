
\section{Introduction}

In typical walkthrough systems, data sets consisting of hundreds of millions of
triangles and many gigabytes of associated data (e.g. walking through a virtual
city) are quite common. Rendering such massive amounts of data requires
out-of-core rendering algorithms that bring only the required data for
rendering into main memory from secondary storage. The time required to accomplish such a task
is known as seek time for our purposes and it
is the time taken to locate the beginning of the required data in the storage
device. \\
\\
In this paper, we store redundant
copies of data in order to reduce the seek time. We propose a model for seek time based on the actual
number of data units
between the requested data units in the linear data layout. Using this model, and given
the data access requirements for a walkthrough application, we develop
an algorithm to duplicate data units strategically to maximize the reduction
in the seek time, while keeping the redundancy factor within the user defined
bound. We show that our greedy solution can generate both the extreme cases
of data layout with redundancy, namely the maximum redundancy case
(a layout where seek time is at most one) and the no-redundancy case (a simple
cache oblivious mesh layout with a potentially high seek time), as well as
reasonable solutions for redundancy factor constraints in between the extremes.
We show that the
implementation of our algorithm significantly reduces average delay and the maximum delay between
frames and noticeably improves the consistency of performance and
interactivity.

\section{Redundancy-based Cache Oblivious Data Layout Algorithm}

\subsection{Definitions}

Let us assume that the walkthrough scene data, including all the levels of
details of the model, are partitioned into equal sized data blocks (say 4KB)
called data units. This is the atomic unit of data that is accessed and fetched
from the disk. Typically, vertices and triangles that are located spatially
closely (and belong to the same level of detail) have high chances of being
rendered together, and hence can be grouped together in a data unit. All the
data units required to render a scene from a viewpoint is labeled as an {\em
access requirement}. There can be many different ways of defining access requirements and data
units.  In general,
the access requirements are determined by the application and are meant to be
sets of data units that are likely to be accessed together. \\
\\
Suppose that we have a linear ordering of data units that may eventually be the
order in which they are stored in the hard drive.  Given an access requirement
$A$, the total span of $A$ is the total number of data units between the first
and last data units that are used by $A$. If a data unit is not required by $A$, but
lies between the first and last unit of $A$, then it is still counted in the
span of $A$. 

\subsection{Seek Time Measure}

Given a linear order of data units and the access requirements, we would like to
estimate the seek time for that application.  For each access requirement, the
read head of the hard disk has to move from the first data block to the last
irrespective of whether the intermediate blocks are read or skipped. Hence the
span of an access requirement can be used as a measure of seek time - time
taken to seek the last data unit starting from the first data unit. We assume all access requirements are equally likely to be used thus 
our total estimated seek time will be as follows:
\[
EST = \sum_{i \in I}{A_i}.
\]
If we reduce the total EST in our optimization, then the average estimated seek time will be reduced. During optimization, we first  choose and process the access requirement with the maximum span. As a result, we not only reduce the average span, but also the maximum span, and hence the standard deviation in spans. This will in turn have an effect of providing consistent rendering performance with low data fetch delays as well as consistently small variation between such delays during rendering. 


\subsection{Algorithm Overview}

Given the access requirements and the data units, the goal of our algorithm is to compute a data layout that reduces EST. In \cite{cacheobliviouslayout}, the only allowed operation on the data units is
the move operation and the optimal layout is computed using only that
operation. For our purposes, we are allowed to copy data units, move them, and
delete them if they are not used. Using these operations, we want minimize EST
while also keeping the number of redundant copies as low as possible. After constructing a cache oblivious layout 
of the data set to get an initial ordering of data units, we copy one data unit
to another location. We reassign one or more of the access requirements that
use the old copy of the data unit to the new copy making sure the EST is
reduced.  If all the access requirements that used the old copy now use the
new copy of the data unit, then the old copy is deleted.  We repeat this
copying and possible deletion of individual data units until our redundancy
limit has been reached. 
