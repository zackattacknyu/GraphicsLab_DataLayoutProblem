
\begin{abstract}

Performance of interactive graphics walkthrough systems depend on the time
taken to fetch the required data from the secondary storage to
main memory. It has been earlier established that a large fraction of this
fetch time is spent on seeking the data on the hard disk. In order to reduce
this seek time, redundant data storage has been proposed in the literature, but
the redundancy factors of those layouts are prohibitively high.  In this paper,
we develop a cost model for the seek time of a layout.  Based on this cost
model, we propose an algorithm that computes a redundant data layout with the
redundancy factor that is within the user specified bounds, while maximizing
the performance of the system. Our data layout method can work with models with textures unlike most other methods. The interactive rendering speed of the walkthrough system was improved by a factor of 2-4 by using our data layout method when compared to existing methods with or without redundancy. \\
\\
Our cost model assumes a linear ordering of the data units and data units which contain one or more access requirements. The seek time is then modeled as the sum of the spans of each of the access requirements. We seek to minimize the seek time. We start by using a heuristic to obtain an ordering of the data that is better than the initial ordering. We then proceed greedily and insert redundant data into the places where it will have the most impact on seek time. \\
\\
This algorithm produces significant advantages by using redundancy and this approach allows us to limit the amount of redundancy that is used to get an optimal seek time. We found this in practice. \\
\\
Our algorithm also allowed us to control the layout and told us exactly where the redundant data units should be placed.\\
\\
Most importantly, our algorithm ran efficiently. It is polynomial in the input size and can thus be leveraged quite well. 


\end{abstract}

