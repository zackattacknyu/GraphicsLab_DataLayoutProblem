
\section{Redundancy-based Cache Oblivious Data Layout Algorithm}


\subsection{Definitions}

Let us assume that the walkthrough scene data, including all the levels of
details of the model, are partitioned into equal sized data blocks (say 4KB)
called data units. This is the atomic unit of data that is accessed and fetched
from the disk. Typically, vertices and triangles that are located spatially
closely (and belong to the same level of detail) have high chances of being
rendered together, and hence can be grouped together in a data unit. All the
data units required to render a scene from a viewpoint is labeled as an {\em
access requirement}. \\
%In order to minimize the number of access requirements, 
\\
There can be many different ways of defining access requirements and data
units. One simple choice is to introduce a concept of navigation space for the
walkthrough application.
The navigation space in the walkthrough scene, which defines the space of all
possible view points, can be partitioned into cells,  and all the data units required by each of the viewpoints within a cell is grouped together to define one access requirement. Thus the
number of cell partitions define the number of access requirements. Primitives
in a data unit can be visible from many viewpoints, and hence that data unit
will be part of many access requirements. \\
\\
That was one example of data units and their access requirements. In general,
the access requirements are determined by the application and are meant to be
sets of data units that are likely to be accessed together. \\
\\
Suppose that we have a linear ordering of data units that may eventually be the
order in which they are stored in the hard drive.  Given an access requirement
$A$, the total span of $A$ is the total number of data units between the first
and last data units that are used by $A$. If a data unit is not required by $A$, but
lies between the first and last unit of $A$, then it is still counted in the
span of $A$. Figure \ref{singleAR} shows a linear order of data units and three
different access requirements shown by solid, double-dashed and dotted lines.
The span of an access requirement is the number of blocks between the first and
the last data unit that use that access requirement. For example, for the
access requirement shown with the solid line, the span is 11; the double-dashed
line one has span 12, and the dotted line one has span 11. A data unit can be
part of many access requirements. In the example shown in Figure \ref{singleAR}, data units
1, 4 and 12 are part of two access requirements and data unit 9 is part of all
three. \\ \\

\begin{figure}[t]
\centering
\includegraphics[width=\columnwidth]{AccessReqsFigure.pdf}
\caption{Illustration of a linear order of data units and three example access
requirements.  Lines connect data blocks that belong to the same access
requirement.  The span of the access requirement shown in the solid line is 11.
%, and represent parts of the span of an access requirement.
}
\label{singleAR}
\end{figure}



\subsection{Seek Time Measure}

Given a linear order of data units and the access requirements, we would like to
estimate the seek time for that application.  For each access requirement, the
read head of the hard disk has to move from the first data block to the last
irrespective of whether the intermediate blocks are read or skipped. Hence the
span of an access requirement can be used as a measure of seek time - time
taken to seek the last data unit starting from the first data unit. In the following measure of total seek time, we use a relative probabilistic measure to include the frequency of use of each access requirement.  Let $I$ be
the set of access requirements and $A_i$ represent the span of the access
requirement $i$. Let $p_i$ be the probability that $A_i$ will be used during rendering. We now define Estimated Seek Time (EST) as:
\begin{equation}
EST = \sum_{i\in I}p_i A_i
\end{equation}
In this paper, we assume all access requirements are equally likely to be used thus all $p_i$ values will be the same. We will use this to simplify the above equation to the following for our purposes.
\[
EST = \sum_{i \in I}{A_i}.
\]
It is important to note that the same measure can be used to describe the data transfer time. As mentioned earlier, whether the data between two required data units is read or skipped, the time taken to go from the first to the last required data unit is a measure of the delay caused by the disk. If all the intermediate data in the span is read, this time will be a measure of data transfer time, and if it is skipped, it is a measure of seek time.  In other words, this measure also defines very well the total data fetch time, which is the sum of data seek and transfer times. However, in this paper, we assume that only the required data is read and use this measure to quantify seek time.  \\
\\
The seek time is also measured in other works \cite{optimizingredundancy,singleseeklayout} as number of seeks and not parameterized using the distance between the required data units. In this work, we model seek time as the distance between the data units and optimize this measure. Using this measure, we show better performance than earlier works.\\
\\
If we reduce the total EST in our optimization, then the average estimated seek time will be reduced. During optimization, we first  choose and process the access requirement with the maximum span. As a result, we not only reduce the average span, but also the maximum span, and hence the standard deviation in spans. This will in turn have an effect of providing consistent rendering performance with low data fetch delays as well as consistently small variation between such delays during rendering. \\
\\
It is interesting to note that \cite{cacheobliviouslayout}
used span to measure the expected number of cache misses.  Typically, with
every cache miss, the missing data will be sought in the disk and fetched, thus
adding to the seek time. In this aspect, using the span to measure the seek time is
justified too.


\subsection{Algorithm Overview}

Given the access requirements and the data units, the goal of our algorithm is to compute a data layout that reduces EST. In \cite{cacheobliviouslayout}, the only allowed operation on the data units is
the move operation and the optimal layout is computed using only that
operation. For our purposes, we are allowed to copy data units, move them, and
delete them if they are not used. Using these operations, we want minimize EST
while also keeping the number of redundant copies as low as possible. After constructing a cache oblivious layout 
of the data set to get an initial ordering of data units, we copy one data unit
to another location. We reassign one or more of the access requirements that
use the old copy of the data unit to the new copy making sure the EST is
reduced.  If all the access requirements that used the old copy now use the
new copy of the data unit, then the old copy is deleted.  We repeat this
copying and possible deletion of individual data units until our redundancy
limit has been reached. \\
\\
{\bf Blocks to Copy:} Note that the span of an access requirement does not
change by moving an interior data unit to another interior location. Cost can
be reduced only by moving the data units that are at the either ends of the access
requirement. This observation greatly reduces the search space of data units to
consider for copying. Additionally, for the sake of simplicity of the algorithm, we operate
on only one data unit at a time. \\
\\
{\bf Location to Copy:} Based on the above observation, given an access
requirement, we can possibly move the beginning or the end data units of an
access requirement to its interior. This will reduce its span, thus reducing
the EST for the layout. However, if the new location of the data unit is in the
span of other access requirements, such as location 11 in Fig.~\ref{singleAR},
it increases the span of each of those accesses (all those three access
requirements in Fig.~\ref{singleAR}) by one unit.
Let $j$ be the new location for the start or end data unit of an access
requirement $i$. Let $\Delta A_i$ denote the change in the span of the access
requirement $i$ by performing this copying operation. Let  $k_j$ denote the
number of access requirements whose span overlaps at location $j$.  The
reduction in EST by performing this copying operation is given by 
%\YOON{We have to use a different name for this equation, since we redefine this
%function in below for the total benefit. It is not a good idea to define two
%times for the same name.}
\[ 
\Delta EST_C(i,j) = \Delta A_i - k_j, 
\]
where $C$ denotes {\it copying} the data unit for access requirement $i$ to the location $j$. We find the location $j$ where the start or end data unit of the access requirement $i$ needs to be copied  using a simple linear search through the span of $i$ as
\[
argmax_j(\Delta EST_C(i,j)).
\]
{\bf Assignment of Copies to Access Requirements:} The above operation would
result in two copies of the same data unit, say $d_{old}$ and $d_{new}$.
Clearly the new copy $d_{new}$ in location $j$ will be used by the access
requirement $i$.  But $d_{old}$ could be accessed by multiple other access
requirements. All other access requirements that accesses $d_{old}$ can either
continue to use $d_{old}$ or use $d_{new}$ depending on the overall effect on
their span. Let $S$ be the set of access requirements whose span does not
increase by using $d_{new}$ instead of $d_{old}$. Now the total benefit by
copying the data unit $d_{old}$ of the access requirement $i$ to the new
location $j$ is
\begin{equation}
\Delta EST_C(i,j) = \Delta A_i - k_j + \sum_{s\in S}\Delta{A_s}.
\label{eq:copyingcost}
\end{equation}
\\
{\bf Moving versus Copying:} Let $T$ be the set of access requirements whose
span will increase by accessing $d_{new}$ instead of $d_{old}$. Further, let
$k_{old}$ be the number of access requirements in whose span $d_{old}$ is. If
we force all the access requirements that uses $d_{old}$ to use $d_{new}$ and
then delete $d_{old}$ -- in other words, if we move $d$ instead of copying --
then the benefit of this move would be given by
\[
\begin{split}
 \Delta EST_M(i,j) & = \Delta A_i - k_j + \sum_{s\in S}\Delta{A_s} + \sum_{t\in
T}\Delta{A_t} + k_{old} \\
% \Delta EST_M(i,j) 
		&= \Delta EST_C(i,j) + \sum_{t\in T}\Delta{A_t} + k_{old},
\end{split}
\]
where $\Delta EST_M(i,j)$ gives the benefit of {\it moving} a start or end data unit of the access requirement $i$ to position $j$. Note that each of $\Delta A_t$ is negative. Hence the benefit of moving might be more or less than the benefit of copying depending on the relative values of $\sum_{t\in T}\Delta{A_t}$ and $k_{old}$. But the main advantage of moving instead of copying is that this operation does not increase the redundancy thus it keeps the storage requirement the same. So we perform moving instead of copying as long as $\Delta EST_M(i,j)$ is positive.\\
\\
{\bf Data Unit processing order:} We now need to figure out how to use this
information to decide in what order the copying and moving should be done. We will make two heaps: $E_M$ and $E_C$. The $E_M$ heap will organize the move operations and consist of the values of $\Delta EST_M(i,j)$ for the start and end data units for all access requirements $i$ where the units are put in their optimal location $j$. The $E_C$ heap will be the same thing except it will organize the copy operations and consist of the values of $\Delta EST_C(i,j)$.\\
\\
We process the $E_M$ heap first as long as the top of the heap is positive and
effect the move of the data unit at the top of the heap. After each removal and
processing, $\Delta EST_M$ and $\Delta EST_C$ of the affected access
requirements and the corresponding heaps are updated. If there are no more data
units where $\Delta EST_M$ is positive, then one element from the top of the
heap $E_C$ is processed. After processing and copying a data unit from the top
of heap $E_C$, the heaps $E_C$ and $E_M$ are again updated with new values for
the affected access requirements. If this introduces an element in the top of
$E_M$ heap with positive values, the $E_M$ heap is processed again. This
process gets repeated until the user defined bound on redundancy factor is
reached. As a summary, the pseudo-code of this algorithm is shown as Algorithm
\ref{pseudocode}.

\begin{algorithm}[t]
Input: Data units and their access requirements (AR) \;
\For{start and end unit of each AR i}{
	Find optimal location $j$ for copy\;
	Calculate $\Delta EST_M(i,j)$ and insert into $E_M$ \;
	Calculate $\Delta EST_C(i,j)$ and insert into $E_C$ \;
}
\While{{\bf true}}{
		\While{top element of $E_M$ is positive}{
			Pop top element and move the data unit to its destination \;
			Update $E_M$ and $E_C$ \;
		}
		\eIf{there is more space for redundancy}{
		Pop top element and copy the data unit to its destination \;
		Update $E_M$ and $E_C$ \;
		}{{\bf break}}
}
\caption{Pseudo-code for our algorithm}
\label{pseudocode}
\end{algorithm}

