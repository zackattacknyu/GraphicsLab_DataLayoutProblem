
\section{Analysis and Comparisons over Prior methods}

In the algorithm, we make a list of data units that will reduce seek time by just moving them. We perform these moves first before doing any copying. This initial step will produce a better solution than proposed by \cite{cacheobliviouslayout} without adding extra units. This is because our algorithm will consider cases where data units are close to each other but in hierarchically different blocks with  the cache oblivious layout. To show this, consider a case where we have two access requirements of 5 data units each. Figure \ref{YoonImprovement} shows an example of that kind of layout. In the middle of that figure is the result of using the cache oblivious layout. Because hierarchically it would arrange the units in each block and then arrange the blocks, it does not detect that the units with the black access requirement can be grouped together. When our algorithm is run, it would find that it can shorten the black access requirements without adding redundancy so it would do that first, as shown in the bottom of that figure.

\begin{figure}[ht]
\centering
\includegraphics[width=3in]{ImprovementOverYoon.jpg}
\caption{Example of two access requirements of 5 data units each. The red line represents the boundary between blocks in the cache oblivious layout hierarchy. The original layout (top), cache-oblvious layout (middle), as well as the layout after running our algorithm (bottom) is shown.}
\label{YoonImprovement}
\end{figure}

Existing algorithms for the data layout problem do not consider redundancy. Even if we find a polynomial time algorithm that solves the data layout problem, we can actually achieve a seek time better than the optimal one without redundancy. We have such an example with figure \ref{fig:startingProb}. As can be seen in the figure, the total seek time is 11 units originally. Without redundancy, the seek time is reduced to 9 units, as found through a brute-force search. With redundancy, the total seek time is the minimum required which is 8 units. While a reduction from 9 to 8 units may not seem dramatic, when this result is scaled up to the hundreds of millions, this makes a big difference in seek time, which we saw in practice.\\

\begin{figure}[ht]
\centering
\includegraphics[width=3in]{examplePic.png}
\caption{Data Units with varying access requirements (top) as well as its optimal layout without redundancy (middle) and its optimal layout with redundancy (bottom)}
\label{fig:startingProb}
\end{figure}

