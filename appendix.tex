\begin{appendix}

\section{Linear Search Justification}

In order to find the best place to copy a data unit, we perform a linear search within the span of the access requirement for location with the largest benefit. If $k$ is the span of the access requirement, then the linear search takes $O(k)$ query time. Updates will also be $O(k)$ time. Construction of the list of data units where each data unit stores the number of overlapping access requirements will be $O(N)$. There are other approaches, such as a range tree or dynamic programming, that may produce better query times, but their construction and update times will be worse as well as their storage. \\
\\
With dynamic programming, we would have to maintain a matrix where an entry
$(i,j)$ would contain the minimum value in that range. This would give us a
$O(1)$ query time but the construction and storage would be $O(N^2)$ where N is
the number of data units. The update time would be $O(N)$ when we add a data
unit. Since the $N$ for this problem domain is in the hundreds of millions,
that is an unacceptable storage bound. The construction run time would also be
prohibitive given the magnitude of our input. \\
\\
We could use a range tree. The initial binary search tree would be sorted by
index and at each entry would be a pointer to a binary search tree sorted by
value. If we put the min value at each of the nodes of the initial tree, we can
speed up our queries. We would get a $O(log N)$ query time, but our
construction time and storage would be $O(N log N)$. Updating the data
structure would take at a minimum $O(k log(N))$ time if we do careful indexing
and only update the nodes that need to be updated. If we have a large access
requirement, then this would represent a significant improvement in query time.
Given our exceptionally large input, however, the construction, storage, and
update bounds are too prohibitive.  \\
\\
As it turns out, the common data structures that would be used for the finding the minimum value in an arbitrary part of a list are not practical for our purposes. Thus, while a simple linear search may seem inefficient at first, as it turns out it is the best option given our constraints. 

\end{appendix}
