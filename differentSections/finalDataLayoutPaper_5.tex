%%% template.tex
%%%
%%% This LaTeX source document can be used as the basis for your technical
%%% paper or abstract. Intentionally stripped of annotation, the parameters
%%% and commands should be adjusted for your particular paper - title, 
%%% author, article DOI, etc.
%%% The accompanying ``template.annotated.tex'' provides copious annotation
%%% for the commands and parameters found in the source document. (The code
%%% is identical in ``template.tex'' and ``template.annotated.tex.'')

\documentclass[conference]{acmsiggraph}

\usepackage{authblk}
\usepackage[]{algorithm2e}

\TOGonlineid{45678}
\TOGvolume{0}
\TOGnumber{0}
\TOGarticleDOI{1111111.2222222}
\TOGprojectURL{}
\TOGvideoURL{}
\TOGdataURL{}
\TOGcodeURL{}

\title{Performance Driven Redundancy Optimization of Data Layouts for Walkthrough Applications}

%\author[1]{Zachary DeStefano\thanks{zdestefa@uci.edu}}
%\author[1]{Shan Jiang\thanks{sjiang1714@gmail.com}}
%\author[1]{Gopi Meenakshisundaram\thanks{gopi.meenakshisundaram@gmail.com}}
%\author[2]{Sung-Eui Yoon\thanks{toinsert}}
%\pdfauthor{Zachary DeStefano,Shan Jiang,Gopi Meenakshisundaram,Sung-Eui Yoon}
\pdfauthor{}
\author{}
%\affil[1]{University of California, Irvine}
%\affil[2]{KAIST}

\keywords{Data Layout Problem, Out-Of-Core Rendering, Cache Oblivious Mesh Layout, Redundant Data Layout, Walkthrough Application}

\begin{document}

%% \teaser{
%%   \includegraphics[height=1.5in]{images/sampleteaser}
%%   \caption{Spring Training 2009, Peoria, AZ.}
%% }

\maketitle

\section{Experimental Results}

\begin{figure}[ht]
  \centering
  \includegraphics[width=3.0in]{city.png}
  \caption{City model: 110 million triangle, 6 GBs. }
  \label{fig:model1}
\end{figure}

\begin{figure}[ht]
  \centering
  \includegraphics[width=3.0in]{boeing.jpg}
  \caption{Boeing model: 350 million triangle, 20 GBs. }
  \label{fig:model2}
\end{figure}

\begin{figure}[ht]
  \centering
  \includegraphics[width=3.0in]{densecity.jpg}
  \caption{Urban model: 100 million triangle, 12 GBs. }
  \label{fig:model3}
\end{figure}


\textbf{Experiment context:}
In order to implement our algorithm, we used a workstation that is a Dell T5400 PC with Intel (R) Core (TM) 2 Quad and $8GB$ main memory. The hard drive is a 1TB Seagate Barracuda with 7200 RPM and the graphics card is an nVIDIA Geforce GTX 260 with 896 MB GPU memory. The data rate of the hard drive is $120$ MB/s and the seek time is a minimum of $2$ ms per disk seek.\\
\\
\textbf{Benchmarks:}
We use three models to perform our experiments, each model represents a use case or scenario. The City model (Figure \ref{fig:model1}) is a regular model that can be used in a navigation simulation application or visual reality walkthough. The Boeing model (Figure \ref{fig:model2}), on the other hand, represents scientific or engineering visualization applications. The Urban model (\ref{fig:model3}) has texture attached to it, which is commonly used in games. By comparing performance of cache-oblivious layout without redundancy to our method using redundancy on these three models, our goal is to show the redundancy based approach can achieve more stable and generally better performance on different real time applications. \\
\\
\textbf{Results:}
Figure \ref{fig:resultall} shows the results of delays caused by fetching data on the experimental models we used. We compare the results of a cache-oblivious layout without redundancy and one with redundancy. For the layout with redundancy, we set the redundancy factor equal to 4.2. It is clear that the performance of the layout with redundancy has generally shorter delays than the cache-oblivious layout without redundancy. As can be observed from the results, although the layout with redundancy does not eliminate delays for most of sample points on the walkthrough path, it reduces delays to a small range and keeps the performance more consistent. This is the benefit we get from using our algorithm which adds redundancy. Since the algorithm tends to eliminate seeks with longer seek time first, in practice the larger delays are avoided.\\
\\

\begin{figure}[ht]
\centering
\includegraphics[width=3.0in]
{resultall.png}
  \caption{Statistics of delays caused by the fetching processes for the Sparse City model (top), the Boeing model (center), and
the Dense City model (bottom), with and without redundancy. }
  \label{fig:resultall}
\end{figure} 

\begin{figure}[ht]
\centering
\includegraphics[width=3.0in]{statistic.png}
  \caption{Plot of the ratio of the EST of layout with redundancy over the EST of cache-oblivious mesh layout withou redundancy. }
  \label{fig:statistic}
\end{figure} 

There is another major benefit to our approach. Since each time we duplicate one data unit, we can halt it when the redundancy factor reaches a certain threshold. This helps us create a data layout with arbitrary redundancy factor without worrying about exceeding the capacity of secondary storage devices. We use this fact to test different redundancy factors and see their results. In Figure \ref{fig:statistic}, we show the results of using layouts with redundancy factors that range from 1.0 to 10.0. The y-axis in this figure is the ratio of the estimated seek time (EST) of the layout with redundancy over the EST of the layout without redundancy. This value starts at 1.0 where redundancy factor is 1.0, meaning no redundancy, and decreases as redundancy factor goes larger. We can see that the rate of this decrement is not constant, and the benefits we gain at beginning are larger than the ones we get later. This implies that most of the performance improvement resides at the earlier phase of raising redundancy factor. This implies that it is worth it to limit the redundancy factor used because after a certain point you are using much more secondary storage space without improving seek time by much. It also implies that our algorithm dramatically reduces seek time in practice by using only small redundancy factors. 


\bibliographystyle{acmsiggraph}
\bibliography{finalPaperRefs}

\end{document}
